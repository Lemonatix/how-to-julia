\documentclass[12pt]{article}
\usepackage[utf8]{inputenc}
\usepackage[T1]{fontenc}
\usepackage[ngerman]{babel}
\usepackage{amsmath,amssymb,amsfonts}
\usepackage{geometry}
\geometry{a4paper, margin=1.5cm}
\usepackage{enumitem}
\usepackage{hyperref}
\setlength{\parskip}{0.5em}
\setlength{\parindent}{0em}

\title{Formelblatt: Analysis}
\author{}
\date{}

\begin{document}

\maketitle

\section{Vollständige Induktion}
\textbf{Prinzip:}
\begin{enumerate}[label=\arabic*.]
  \item \textbf{Induktionsanfang:} Zeige \(P(n_0)\) für den Startwert \(n_0\) (z.B. \(n_0=0\) oder \(1\)).
  \item \textbf{Induktionsvoraussetzung:} Angenommen, \(P(n)\) gilt für ein beliebiges, aber festes \(n \ge n_0\).
  \item \textbf{Induktionsschritt:} Beweise, dass \(P(n) \Rightarrow P(n+1)\).
\end{enumerate}

\section{Rechenregeln}
\subsection*{Wurzeln:}
\[
\sqrt{ab} = \sqrt{a}\sqrt{b}, \quad \sqrt{\frac{a}{b}} = \frac{\sqrt{a}}{\sqrt{b}}, \quad \sqrt[n]{a^m} = a^{\frac{m}{n}}.
\]
\subsection*{Logarithmen:}
\[
\log(ab) = \log a + \log b, \quad \log\left(\frac{a}{b}\right) = \log a - \log b, \quad \log(a^r)=r\log a.
\]
\subsection*{Potenzen:}
\[
a^{m+n} = a^m \cdot a^n, \quad (a^m)^n = a^{mn}, \quad a^{-n} = \frac{1}{a^n}.
\]

\section{Komplexe Zahlen}
\subsection*{Kartesische Darstellung:}
\[
z = x + iy,\quad x,y \in \mathbb{R}.
\]
\subsection*{Polardarstellung:}
\[
z = r (\cos\varphi + i \sin\varphi),\quad r=|z|,\quad \varphi=\arg(z).
\]
\subsection*{Euler-Formel:}
\[
e^{i\varphi} = \cos\varphi + i \sin\varphi.
\]

\section{Additionstheoreme}
\subsection*{Trigonometrisch:}
\[
\sin(\alpha \pm \beta) = \sin\alpha \cos\beta \pm \cos\alpha \sin\beta,
\]
\[
\cos(\alpha \pm \beta) = \cos\alpha \cos\beta \mp \sin\alpha \sin\beta.
\]
\subsection*{Hyperbolisch:}
\[
\sinh(\alpha \pm \beta) = \sinh\alpha \cosh\beta \pm \cosh\alpha \sinh\beta,
\]
\[
\cosh(\alpha \pm \beta) = \cosh\alpha \cosh\beta \pm \sinh\alpha \sinh\beta.
\]

\section{Binomialkoeffizient}
\subsection*{Ohne Wiederholung (Kombinationen):}
\[
\binom{n}{k} = \frac{n!}{k!(n-k)!}, \quad 0 \le k \le n.
\]
\subsection*{Mit Wiederholung:}
\[
\binom{n+k-1}{k} = \frac{(n+k-1)!}{k!(n-1)!}.
\]

\section{Grenzwerte}
\subsection*{Definition:}
\[
\lim_{n \to \infty} a_n = L \quad \Longleftrightarrow \quad \forall \varepsilon > 0 \; \exists N \in \mathbb{N} : |a_n - L| < \varepsilon \quad \forall n \ge N.
\]
\subsection*{L'Hôpital's Regel:}
Falls \( \lim_{x \to a} f(x) = \lim_{x \to a} g(x) = 0\) oder \(\pm\infty\) und \(g'(x) \neq 0\) in einer Umgebung von \(a\), so gilt:
\[
\lim_{x \to a} \frac{f(x)}{g(x)} = \lim_{x \to a} \frac{f'(x)}{g'(x)},
\]
sofern der Grenzwert auf der rechten Seite existiert.

\section{Sätze zu Folgen}
\begin{itemize}
  \item Jede konvergente Folge ist beschränkt.
  \item Jede monotone und beschränkte Folge konvergiert.
  \item \textbf{Bolzano-Weierstraß:} Jede beschränkte Folge besitzt mindestens eine konvergente Teilfolge.
\end{itemize}

\section{Konvergenzkriterien für Reihen und Potenzreihen}
\subsection*{Leibnizkriterium (Alternierende Reihen):}
Ist \( (a_n) \) monoton fallend und \( \lim_{n\to\infty} a_n = 0 \), dann konvergiert
\[
\sum_{n=1}^\infty (-1)^{n-1} a_n.
\]
\subsection*{Nullfolgenkriterium:}
Falls \( \sum_{n=1}^\infty a_n \) konvergiert, so gilt \( \lim_{n\to\infty} a_n = 0 \).
\subsection*{Absolute Konvergenz:}
\[
\text{Die Reihe } \sum_{n=1}^\infty a_n \text{ konvergiert absolut, wenn } \sum_{n=1}^\infty |a_n| \text{ konvergiert.}
\]
\subsection*{Majorantenkriterium:}
Sind \( |a_n| \le b_n \) für alle \( n \) und konvergiert \( \sum_{n=1}^\infty b_n \), so konvergiert \( \sum_{n=1}^\infty a_n \) absolut.
\subsection*{Quotientenkriterium:}
Sei
\[
L = \lim_{n\to\infty} \left|\frac{a_{n+1}}{a_n}\right|.
\]
Dann:
\[
L < 1 \Rightarrow \text{absolute Konvergenz}, \quad L > 1 \Rightarrow \text{Divergenz}, \quad L = 1 \Rightarrow \text{unbestimmt.}
\]
\subsection*{Wurzelkriterium:}
Sei
\[
L = \limsup_{n\to\infty} \sqrt[n]{|a_n|}.
\]
Dann:
\[
L < 1 \Rightarrow \text{absolute Konvergenz}, \quad L > 1 \Rightarrow \text{Divergenz}, \quad L = 1 \Rightarrow \text{unbestimmt.}
\]
\subsection*{Umordnungssatz:}
Absolute Konvergenz garantiert, dass jede Umordnung den Wert der Reihe nicht ändert.

\section{Grenzwerte von Funktionen \& Stetigkeit}
\subsection*{Zwischenwertsatz:}
Sei \( f:[a,b] \to \mathbb{R} \) stetig und \( f(a) \neq f(b) \). Dann gilt:
\[
\forall c \text{ zwischen } f(a) \text{ und } f(b) \; \exists \xi \in [a,b] : f(\xi) = c.
\]
\subsection*{Ableitung der Umkehrfunktion:}
Ist \( f \) streng monoton und differenzierbar, so existiert \( f^{-1} \) mit
\[
(f^{-1})'(y) = \frac{1}{f'(x)} \quad \text{mit } y=f(x).
\]
\subsection*{Extremwertsatz:}
Eine stetige Funktion auf einem kompakten Intervall \( [a,b] \) nimmt ihr Maximum und Minimum an.

\section{Differenzialquotient und Ableitungen}
\subsection*{Definition der Ableitung:}
\[
f'(x) = \lim_{h \to 0} \frac{f(x+h)-f(x)}{h}.
\]
\subsection*{Ableitungsregeln:}
\begin{itemize}
  \item \textbf{Summenregel:} \( (f+g)'(x) = f'(x) + g'(x) \).
  \item \textbf{Produktregel:} \( (fg)'(x) = f'(x)g(x) + f(x)g'(x) \).
  \item \textbf{Quotientenregel:} \( \left(\frac{f}{g}\right)'(x) = \frac{f'(x)g(x)-f(x)g'(x)}{[g(x)]^2} \).
  \item \textbf{Kettenregel:} \( (f\circ g)'(x) = f'(g(x)) \cdot g'(x) \).
  \item \textbf{Ableitung der Umkehrfunktion:} Siehe Zwischenwertsatz.
\end{itemize}

\section{Extremstellen}
\subsection*{Notwendige Bedingung:}
Falls \( f \) an \( x_0 \) ein lokales Extremum hat und differenzierbar ist, so gilt \( f'(x_0)=0 \).
\subsection*{Hinreichende Bedingung:}
Ist \( f'(x_0)=0 \) und
\[
f''(x_0) > 0 \quad (\text{lokales Minimum}) \quad \text{oder} \quad f''(x_0) < 0 \quad (\text{lokales Maximum}),
\]
so liegt ein Extremum vor.
\subsection*{Sattelpunkte:}
Falls \( f'(x_0)=0 \) und \( f''(x_0)=0 \), sind höhere Ableitungen zu untersuchen.

\section{Satz von Rolle und Mittelwertsatz}
\subsection*{Satz von Rolle:}
Sei \( f:[a,b]\to\mathbb{R} \) stetig auf \( [a,b] \) und differenzierbar auf \( (a,b) \) mit \( f(a)=f(b) \). Dann existiert ein \( \xi \in (a,b) \) mit
\[
f'(\xi)=0.
\]
\subsection*{Mittelwertsatz:}
Ist \( f:[a,b] \to \mathbb{R} \) stetig und auf \( (a,b) \) differenzierbar, so existiert ein \( \xi \in (a,b) \) mit
\[
f'(\xi) = \frac{f(b)-f(a)}{b-a}.
\]

\section{Taylorreihen}
\subsection*{Satz von Taylor:}
Für \( f \) in einer Umgebung von \( a \) gilt:
\[
f(x) = \sum_{k=0}^{n} \frac{f^{(k)}(a)}{k!}(x-a)^k + R_n(x),
\]
wobei \( R_n(x) \) der Restterm ist.

\section{Integrale}
\subsection*{Hauptsatz der Integralrechnung:}
Sei \( f \) stetig auf \( [a,b] \) und \( F \) eine Stammfunktion von \( f \) (\( F' = f \)). Dann:
\[
\int_a^b f(x)\,dx = F(b)-F(a).
\]
\subsection*{Substitution:}
Mit \( u=g(x) \) gilt:
\[
\int f(g(x))g'(x)\,dx = \int f(u)\,du.
\]
\subsection*{Partielle Integration:}
\[
\int u(x)v'(x)\,dx = u(x)v(x) - \int u'(x)v(x)\,dx.
\]
\subsection*{Riemannscher Integralbegriff:}
Definition über Zerlegung des Integrationsintervalls, Riemann-Summen und Grenzwertbildung.
\subsection*{Cauchyscher Hauptwert:}
Für symmetrische uneigentliche Integrale:
\[
\text{p.v.} \int_{-a}^{a} f(x)\,dx = \lim_{a \to \infty} \int_{-a}^{a} f(x)\,dx.
\]
\subsection*{Uneigentliche Integrale:}
\[
\int_a^\infty f(x)\,dx = \lim_{b \to \infty} \int_a^b f(x)\,dx.
\]
\subsection*{Integralkriterium für Reihen:}
Sei \( a_n = f(n) \) mit \( f \) monoton fallend und \( f(x)\ge0 \). Dann gilt:
\[
\sum_{n=1}^\infty a_n \text{ konvergiert } \Longleftrightarrow \int_1^\infty f(x)\,dx < \infty.
\]

\section{Differentialgleichungen (DGL)}
\subsection*{Erste Ordnung}
\subsubsection*{Trennung der Variablen:}
\[
\frac{dy}{dx} = f(x)g(y) \quad \Rightarrow \quad \int \frac{dy}{g(y)} = \int f(x)\,dx.
\]
\subsubsection*{Lineare DGL:}
\[
y' + p(x)y = q(x), \quad \mu(x)=\exp\left(\int p(x)\,dx\right),
\]
Lösung:
\[
y(x)=\frac{1}{\mu(x)}\left( \int \mu(x)q(x)\,dx + C \right).
\]

\subsection*{Zweite Ordnung}
\subsubsection*{Homogene DGL:}
\[
a\, y'' + b\, y' + c\, y = 0, \quad \Delta = b^2 - 4ac.
\]
Lösungen:
\[
\begin{cases}
\Delta > 0: & y = C_1 e^{\lambda_1 x} + C_2 e^{\lambda_2 x}, \quad \lambda_{1,2}=\frac{-b\pm\sqrt{\Delta}}{2a},\\[1ex]
\Delta = 0: & y = (C_1 + C_2 x)e^{\lambda x}, \quad \lambda=-\frac{b}{2a},\\[1ex]
\Delta < 0: & y = e^{\alpha x}\left(C_1 \cos(\beta x) + C_2 \sin(\beta x)\right), \quad \alpha=-\frac{b}{2a},\; \beta=\frac{\sqrt{-\Delta}}{2a}.
\end{cases}
\]
\subsubsection*{Inhomogene DGL:}
Gesamtlösung:
\[
y(x)= y_h(x) + y_p(x),
\]
wobei \(y_h\) die Lösung der homogenen Gleichung und \(y_p\) eine spezielle Lösung der inhomogenen DGL ist.  
Methoden: Variation der Konstanten oder Ansatz der unbestimmten Koeffizienten.

\end{document}